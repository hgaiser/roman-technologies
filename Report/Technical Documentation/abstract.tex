\documentclass[project_eva.tex]{subfiles}
\begin{document}

Eva\textquotesingle s aesthetic design is based on the final concept drawing that was made. The model was split into two parts of different complexity: the head part and the body part. The head was redesigned foremost based on the demands of emotion display. This means that the design had to be thin enough to allow the RGB LEDs to shine through and allow placement of eyebrows that could move vertically and rotational. Assembling the head was difficult because of the thin material. The main design consideration for the main body was the usage of the mobile base that was delivered by the minor organisation. The covers were made using a thermoforming technique because of the low costs and the availability of the technique in the machine hall in the faculty of Industrial Design. Due to time constraints we have decided not to colour her yet and due to the contrast with the TU Delft logo it is decided that Eva will not be coloured at all. Eva uses emotional states to communicate; the emotion states of Eva are constructed using lights, sounds and motion. The emotional states are related to the states Ekman (1982) described, which are mostly expressed by the head. To design the head, some tradeoffs between the visual appearance and the mechanical part had to be made. The head was designed in such a way that it is not only a cover, but also the frame everything is connected to. 

To perform fetch and carry tasks, Eva needed an arm. The main idea is very simple. The upper arm can move up and down, like a simplified shoulder. Further down the arm is the elbow, whick makes sure the lower arm is always horizontal. The last part of the arm can move sideways. As the motors driving the arm have relative encoders, they need a reference point to know at which angle the arm is positioned. This is implemented by a switch on each degree of freedom. Eva\textquotesingle s arm was designed with a focus on minimizing production time. Small shafts or connectors were mostly created from steel, since they generally have high stress concentrations on them. Furthermore, due to their small dimensions the extra weight is minimal. The assumptions about material choice have been validated using numerous calculations. 

Eva\textquotesingle s electrical system consists of three subsystems. First there is the power electronics part built
around the battery and the DC power input. This provides the necessary power to the other subsystems: the motor electronics and the mainboard. Eva can be powered either from a 12V battery or a 12V DC power input. She has a control panel behind the front bumper from where you can switch between battery power and net power. The emergency stop affects the wheel and arm motors. When the emergency button is pushed the relays will cut the power to the motors and the 3mxl is signaled to stop driving the motors as well. Eva sports a total of seven motors and three servos; two Dynamixels actuate the head. Besides the motors others parts have to be controlled as well. Eva’s head is full of Electronics: three servo\textquotesingle s and 24 RGB LEDs. Eva is also geared up with 10 Ultrasonic sensors around the base, four bumper sensors and the ultrasonic Ping sensor at the gripper. All the electronics necessary to control these parts are combined in the main board located in the back of Eva.

Eva\textquotesingle s software is implemented in C++, using ROS. In the implemented software structure, each package represents a physical part of Eva, with a logical unit package acting as her ``brain´´, which receives data from Eva\textquotesingle s sensors and encoders and which makes decisions based on the readings. 

In order for Eva to move to her destination, a path planner was needed. For this purpose, Recast was used, but a conversion from a 2D map to a 3D one needed to be made. When the path is planned, the path handler will use it to move Eva along the path. The path handler works almost as a spring would, where the spring would follow the ideal path calculated. Object avoidance is done using ultrasone sensors. Depending on where an object is detected
with the ultrasone sensors, the speed of the left or right wheel will be scaled down, so Eva will avoid the object.













\end{document}