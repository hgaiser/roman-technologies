\documentclass[project_eva.tex]{subfiles}
\begin{document}
Eva is a service robot that performs activities of daily living, designed for making elderly more self-
sufficient\cite{Forlizzi} and home-care professionals more efficient in their distribution of time across clients. These 
activities are based on the tasks of helper dogs that have a proven efficiency on the reduction of the need for professional 
healthcare help\cite{Diepenhorst}. Eva improves the service of helper dogs by allowing expansion into ``care at distance'' 
\cite{Evers} through digital means (for example, video conversations through a screen) and by allowing the improvement of the 
tasks performed by an ADL dog by losing the restraints posed by a dog’s body and mind. 

Eva has a set of three emotional states: happy, sad and novelty, that serve as communication methods. Eva communicates these 
states by non-textual vocal sounds, eyebrow movement and lights \footnote{See Technical Document}. These elements allow 
communication with her even if the vision or hearing of the user is impaired – which is often the case with our target group, 
due to ailments of old age.  All elements of her communication are kept basic to allow her to use a very basic method of 
emotional communication. This basic form of the states  also allows Eva to communicate a large amount of messages using only 
a small array of emotions. This is because the states can be interpreted depending on the context of the action. 

\end{document}