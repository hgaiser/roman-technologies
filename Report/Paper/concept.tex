\documentclass[project_eva.tex]{subfiles}
\begin{document}
Eva should be a service robot that performs activities of daily living, designed for making elderly more self-
sufficient\cite{Forlizzi} and home-care professionals more efficient in their distribution of time across clients. These 
activities are based on the tasks of helper dogs that have a proven efficiency on the reduction of the need for 
professional healthcare help\cite{Diepenhorst}. In order to perform such tasks, Eva needs tools to navigate in the 
environment, for object manipulation and to interact with the user. Some of these tools were already supplied for this 
project (gripper for object manipulation \footnote{See Assignment one of this project}, mobile base for navigating 
\footnote{See Assignment two of this project}. In addition, an arm needs to be built for Eva for object manipulation. This 
arm should have a good range to reach for objects and should be able to be tucked in when Eva is driving around for safety 
reasons. Also, Eva should improve the service of helper dogs by allowing expansion into ``care at distance'' \cite{Evers} 
through digital means (for example, video conversations through a screen) and by allowing the improvement of the tasks 
performed by an ADL dog by losing the restraints posed by a dog’s body and mind.

Eva should have a set of three emotional states \pageref{sec:Emotion expression}: happy, sad and novelty, that serve as 
communication methods. Eva will then communicate these states by non-textual vocal sounds, eyebrow movement and lights 
\footnote{See Technical Document}. This because these elements allow communication even if the vision or hearing of the 
user is impaired - which is often the case with our target group, due to ailments of old age. All elements of her 
communication should be kept basic to allow her to use a very primitive method of emotional communication, making it 
obvious for the user what she is trying to tell. This basic form of states also allows Eva to communicate a large amount of 
messages using only a small array of emotions, this is because the emotional states can be interpreted depending on the 
context of the action.

Eva should understand the feedback of users by determining the emotional load of their reactions. She should then return 
feedback herself to the user by showing emotional states - showing that she will adapt her behavior in a positive or 
negative manner. This allows Eva to adapt her complete behavior to the user.

Eva should learn actively and passively. The method mentioned before is passive learning. With passive learning Eva 
improves her behaviour directly after the accomplishment of her tasks. She should for instance learn to bring coffee every 
time the user asks for a drink at a certain time. Obviously, Eva should also learn to take other factors into consideration, such as the time a command is given (The user probably won't ask for coffee at night, but would want to have water instead).

\label{sec:Concept}
Active learning is the second learning form. For active learning one could imagine Eva learning new (personal) objects so 
she can help the user with personal items that are not available in the library of general objects that Eva does recognize. 
For this form of learning the user should actively interact with Eva, showing her new items, locations, etc. Eva should 
learn actively by combining the ``task'' with the ``object''. 

\end{document}