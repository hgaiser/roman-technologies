\documentclass[project_eva.tex]{subfiles}
\begin{document}
Eva should be a service robot that performs activities of daily living, designed for making elderly more self-
sufficient\cite{Forlizzi} and home-care professionals more efficient in their distribution of time across clients. These 
activities are based on the tasks of helper dogs that have a proven efficiency on the reduction of the need for 
professional healthcare help\cite{Diepenhorst}. Eva improves the service of helper dogs by allowing 
expansion into ``care at distance'' \cite{Evers} through digital means and by allowing the improvement of the tasks 
performed by an ADL\footnote{Activities of Daily Living} dog by losing the restraints posed by a dog\textquotesingle s body and mind: Eva\textquotesingle s design is more flexible than the 
design of a dog. An example: a dog cannot bring hot coffee, while Eva can.

Eva should have a set of three emotional states \pageref{sec:Emotion expression}: happy, sad and novelty, which serves as 
communication methods. Eva will then communicate these states by non-textual vocal sounds, eyebrow movement and lights 
\footnote{See Technical Document}. These elements allow communication even if the vision or hearing of the 
user is impaired - which is often the case with our target group, due to ailments of old age. All elements of her 
communication should be kept basic to allow her to use a very comprehensible form of emotional communication, making it 
obvious for the user what she is trying to tell. This basic form of states also allows Eva to communicate a large amount of 
messages using only a small array of emotions, this is because the emotional states can be interpreted depending on the 
context of the action.

Eva should understand the feedback of users by determining the emotional load of their reactions. She should then return 
feedback herself to the user by showing emotional states - showing that she will adapt her behaviour in a positive or 
negative manner. This allows Eva to adapt her complete behaviour to the user.

Eva should learn actively and passively. The method mentioned before is passive learning. With passive learning Eva 
improves her behaviour directly after the accomplishment of her tasks. This is achieved by having the user give positive 
feedback on ``bringing coffee (a specific drink)''  when the original command was ``get me a (random) drink''.  Eva will 
then slowly change her behaviour in such a way that she will learn to ignore other potential drinks – and in the end will 
only bring coffee when asked for drinks. Using the same method of passive learning, she could also learn to bring coffee 
only in the morning or in the afternoon.

\label{sec:Concept}
Active learning is the second learning form. For active learning one could imagine Eva learning new objects so 
she can help the user with personal items that are not available in the library of objects that Eva does recognize. 
For this form of learning the user should actively interact with Eva, showing her new items, locations, etc. Eva should 
learn actively by combining the ``task'' with the ``object''. An example: the task ``get'' is always known, but it will 
result in an active learning process if the object is unknown. If the object is known, Eva simply performs the task and 
retrieves the object. The same can be done for other tasks involving for instance locations instead of objects.

\end{document}