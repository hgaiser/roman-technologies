\documentclass[project_eva.tex]{subfiles}
\begin{document}
The actual production of a robot that can fully support people with Activities of Daily Living(ADL)\footnote{basic tasks of everyday life, such as eating, bathing, dressing, toileting, and transferring}  tasks was a too large assignment for the minor time period. The project was narrowed down to providing working fetch-and-carry functionality; a suitable interaction method for interacting with humans. The way these were implemented are based on the full concept of the home care robot. This concept is presented below.

The robot should be a service robot that performs activities of daily living, designed for making elderly more selfsufficient\cite{Forlizzi} and home-care professionals more efficient in their distribution of time across clients. These 
activities are based on the tasks of helper dogs that have a proven efficiency on the reduction of the need for 
professional healthcare help\cite{Diepenhorst}. The robot improves the service of helper dogs by allowing 
expansion into ``care at distance'' \cite{Evers} through digital means and by allowing the improvement of the tasks 
performed by an ADLdog. A robot can be built specifically for the ADL tasks, while a dog is not born to do ADL tasks. So a robot could eventually outperform a dog.  A simple example of the biological constraints a robot would not have is that a dog cannot reach all the desired places in a home, such as high cabinets.

Because humans communicate affectively with each other, and often do so with technology, to facilitate interacting with the robot, it should give affective feedback. At the very least, it should be able to detect positive and negative feedback with regards to its behaviour, as well as communicate to the user in a clear way that it will adapt her behaviour or understood a command from the user.

The robot should learn by feedback and learn by guidance \cite{Thomaz} . Learning by feedback is learning by adjusting the robot\textquotesingle s behaviour based on the intentional information of users after the robot has completed a task. The second method is learning by guidance, where the intentional information is received before the fact. Learning by feedback is used for adjusting the robot\textquotesingle s ``known'' behaviour. An example of this method is the robot learning to bring coffee when asked for drinks. This can be achieved by using the intentional negative or positive information from users.  This specific feedback-method is also called reinforcement learning: ``Where an agent learns behaviour by exploring an environment and learning which action to repeat and avoid based on positive and negative feedback respectively'' \cite{Joost}. 
Learning by guidance allows the robot to learn before actually performing a task. For example, it allows the robot to learn recognize new objects to adapt them to the robot\textquotesingle s known tasks. The robot can use this learning method by having two levels of command recognition: task and object. The task part refers to command-parts such as ``get'' or ``go to'', whereas the object part refers to command-parts such as ``coffee'' or ``kitchen''. The tasks are preprogrammed, while the objects can be learned on the go. This allows the robot to adapt to its user by learning to apply its tasks to new (personal) areas, items, etc. 

This was the concept of the robot: how it should work. Eva is the prototype robot built for this project that is based on this concept.

\end{document}