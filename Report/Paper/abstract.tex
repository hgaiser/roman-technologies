\documentclass[project_eva.tex]{subfiles}
\begin{abstract}
This paper focuses on the design of a robot prototype that
is autonomously able to support people in the domestic health care sector. Because of time constraints, the project was narrowed down to providing working fetch-and-carry functionality, navigation, object avoidance and affective interaction methods. Eva is able to recognize emotions in the words spoken by the user; Eva can also understand commands. To communicate with the user, she uses emotional states. The results of her performance are as expected, but there are limitations on the amount of change in the environment where Eva operates. The most important change to be made to the prototype, in order to use Eva in the health care, is to stabilize Eva's performance under different circumstances. 

\end{abstract}