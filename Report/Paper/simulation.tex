\documentclass[project_eva.tex]{subfiles}
\begin{document}

Eva performed  (parts) of her tasks in two different situations: in the simulated home environment of the faculty of Industrial Design and at the ``Hightech meets Design''  symposium. The entire scenario for which Eva was built was tested in the home environment. This environment was chosen because it is very similar to the actual environment in which Eva should operate. Secondly, Eva performed parts of her tasks during the symposium, where she also interacted with symposium. Both situations were considered tests and Eva's behavior was observed.

The test to validate the entire scenario of Eva can be categorized as a ``fetch and carry'' test. To start, a person calls for Eva to get her attention. Then the person asks her to bring a drink. When Eva has processed the voice commands, she navigates to the table where the drink is located. At the table Eva searches for the desired object and tries to grab it. When she cannot see the object, she will turn her head to right and left to see if the object is present. If she does find the object at an angle, she will turn her body towards it and proceed the process. She will try three times to grab the object. Finally, she navigates back to the person to hand over the drink if she got it, otherwise she will return sad. After this, the user is able to give Eva feedback. During this process Eva will at her turn respond with her own emotions.The fetch and carry test in an ideal domestic environment is shown in this video\ref{VIDEO}. 

Children are the least influenced by external experiences so that they can be used most reliably to test �the complete package of the emotional states�.  At the symposium Eva interacted with one child; this is the base for the findings around interaction.
 


\end{document}