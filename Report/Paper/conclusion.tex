\documentclass[project_eva.tex]{subfiles}
\begin{document}

Eva performs her fetch and carry scenario as expected. However, there are limitations on the amount of change in the 
environment where Eva operates. All findings show that Eva operates correctly in situations with the correct 
parameters. However, if one of the parameters changes, the performance of Eva will deteriorate. For instance, if Eva is 
driving on a carpet instead of on a wooden floor, then she will slip more, making her drive inaccurate. Another example 
is that Eva cannot distinguish multiple faces when a lot of people stand around her, or that she cannot recognize the 
command well if ambient noice is present. When these limitations are surpassed issues can occur like the ones described 
in \pageref{simulation} .

Affect detection also worked as expected, but it could be improved with simple smile detection, since it is not hard to 
detect smiles on human faces (look for white teeth and lifted mouth corners \cite{autosmiley} \cite{smile} ). For future work, a bimodal approach \cite{bimodal} for affect detection could be used to improve Eva\textquotesingle s ability in 
distinguishing different levels of affection. An extension to this would be emotion recognition to further improve 
interaction. The most important change to be made to the prototype, in order to use Eva in the health care, is to stabilize Eva\textquotesingle s performance under different circumstances. Also, Eva should be able to perform tasks other than fetch and carry, so she can be of great help to her owner. Last but not least, an extension to Eva would be learning behaviour -as mentioned in the concept \pageref{concept} . This was not implemented due to time constraints. We consider Eva as proof -despite of the many improvements to be made, that robots can be used in the near future in the health care section.
\end{document}
